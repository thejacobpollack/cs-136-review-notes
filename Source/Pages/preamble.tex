\section*{Preamble}
\addcontentsline{toc}{subsubsection}{Preamble}

In the second edition (\docbuild) of my review notes you will find a further in depth perspective (of my original perspective) on the material presented during the University of Waterloo's summer semester offering of CS 136. These notes are not aimed to replace lectures, tutorials, labs, office hours with your professor or TA, Piazza, or any other source directly provided by the course. It is solely aimed as a study aid that you can refer to for peer perspective on the course material.\\

These notes were written during the University of Waterloo's summer semester offering of CS 136 in 2013. It covers only midterm material (modules 1 - 6) and was written pre-midterm as a review for the midterm itself. The material that influenced the contents these notes are as follows:

\begin{itemize}

\item Lecture slides designed by the professors.
\item "C Programming a Modern Approach -- Second Edition", textbook for the course.
\item Class averages on assignments (publicly posted on Piazza).
\item Other posts on the CS 136 Piazza class.
\item Student perspectives.
\item and other third party websites used to improve our understanding.

\end{itemize}

To preserve the chronological order that the material was presented during lecture, these review notes will follow a similar structure. However, some material in these review notes may be introduced earlier or later than with respect to the lectures. This will only occur if we feel as if material flows better in a separate section than it was originally introduced in.\\

Practice exercises can be found scattered throughout each section of the review notes. There is a special section at the end dedicated to sample midterm problems based on emphasized course material, struggling assignment questions and other factors.

\subsubsection*{Prerequisites}

\textit{Student: Can you give me some pointers for the CS 136 midterm?}\\
\textit{Professor: 0xff98120f, 0x000000ff, 0x89ff7192, ...}\\

It is expected that before you begin reading these notes that you have attended every lecture, completed assignments 1 through 5 (and possibly 6) and that you have already been introduced to a majority the material presented in these notes.\\

You should be comfortable using RunC with gedit as your editor for your Racket and C programs. In addition, possibly the most important prerequisite, you should have at least a 4 day window before your midterm while you are looking through these notes. It will be extremely hard to power through these carefully designed review notes and understand all the concepts the night before the midterm.